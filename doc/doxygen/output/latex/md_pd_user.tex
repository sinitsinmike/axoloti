\{\#pd\+\_\+user\}\hypertarget{md_pd_user_pd_user_conversion}{}\section{Pure Data conversion}\label{md_pd_user_pd_user_conversion}
\hypertarget{md_pd_user_pd_user_scaling}{}\subsection{Scaling, units}\label{md_pd_user_pd_user_scaling}
PD (Pure Data) is scaled in physical units represented in floating point numbers\+: Hertz (Hz) for frequency, milliseconds for time. Axoloti parameters have a fixed point number representation that is closer to the implemented algorithms. Axoloti parameters have a range of 0.\+0u to 64.\+0u or -\/64.\+0u to 64.\+0u. For math this represents the 0.\+0 to 1.\+0 or -\/1.\+0 to 1.\+0 range. That means 64.\+0u multiplied with 64.\+0u equals 64.\+0u. For trigonometric functions, -\/64.\+0u corresponds to -\/180 degrees, 64u to 180 degrees. For pitch frequency, 0.\+0u corresponds to 329.\+6\+Hz which is the standard tuning of the middle \char`\"{}\+E\char`\"{} note on a music keyboard. One unit increment is the next key higher on the keyboard (semitones). 12.\+0u up is a doubling of frequency (octave interval in music). 12.\+0u down is half the frequency.\hypertarget{md_pd_user_pd_user_control_rate_vs_messages}{}\subsection{Control rate versus messages}\label{md_pd_user_pd_user_control_rate_vs_messages}
PD uses messages (events) to trigger the execution of methods of objects. Messages can be a single number, a symbol, or a list of numbers and symbols. The PD message architecture needs dynamic memory allocation which conflicts with running on architectures with limited memory. The messages also need to be parsed, preventing inter-\/object compiler optimization. Axoloti does not have such messages, to avoid spending computation time on message parsing. The \char`\"{}bang\char`\"{} message to trigger parameter-\/less methods of objects in PD is often replaced with a \char`\"{}trig\char`\"{} (trigger) input in Axoloti. A \char`\"{}trig\char`\"{} input is activated on the transition from false to true. Axoloti does not have \char`\"{}hot\char`\"{} or \char`\"{}cold\char`\"{} inlets. Execution in Axoloti is not message-\/driven. Execution order is as described in the chapter \char`\"{}\+Execution order\char`\"{}. This means there is no equivalent of these PD objects\+: 
\begin{DoxyItemize}
\item trigger (t) 
\item bang 
\item loadbang 
\item select (s) 
\item route (r) 
\item moses 
\item spigot 
\item swap 
\item change 
\item pack 
\item unpack 
\item until 
\end{DoxyItemize}\hypertarget{md_pd_user_pd_user_wire_types}{}\subsection{Wire types}\label{md_pd_user_pd_user_wire_types}
PD has thick black cables for audio rate (s-\/rate) signals. Axoloti uses red cables and connectors for this purpose. PD implicitly adds multiple audio signals connected to a single input. In Axoloti you have to use an explicit + object, or a mixer. PD has thin black cables for messages. Messages do not exist in Axoloti. Axoloti has control-\/rate signals instead.\hypertarget{md_pd_user_pd_user_object_arguments}{}\subsection{Object arguments}\label{md_pd_user_pd_user_object_arguments}
Axoloti does not have object arguments.\hypertarget{md_pd_user_pd_user_object_conversions}{}\subsection{Object conversions (\+P\+D to Axoloti)}\label{md_pd_user_pd_user_object_conversions}

\begin{DoxyItemize}
\item osc$\sim$ \+: The axoloti osc/square does not have a frequency input in Hz. Adjust the pitch dial to set a constant frequency. If a changing frequency is needed, the \char`\"{}pitchm\char`\"{} inlet can be used to modulate the pitch. 1.\+0u on this input corresponds with increasing the pitch dial one unit. 
\item metro \+: The axoloti lfo/square object outputs a periodic square wave signal. This can be connected to yellow \char`\"{}trig\char`\"{} inputs. The cycle time in milliseconds can be found after clicking 2 times on the frequency label to change its unit to time. 
\item line/vline \+: The Axoloti env/line objects produce 2-\/interval or 3-\/interval piecewise linear ramps with adjustable time intervals and values. 
\item f or float \+: With the hot and cold inputs, the float object can be used as a memory. 
\item sig$\sim$ \+: use nointerp$\sim$ or interp$\sim$ to convert a control signal to audio rate, without or with linear interpolation. 
\item select, route \+: while not directly equivalent (event-\/driven), the inmux objects can be used to select/route one of its inputs. 
\item catch$\sim$, throw$\sim$ \+: no equivalent yet. Use wires and mixers. 
\item send$\sim$ (s$\sim$), receive$\sim$ (r$\sim$) \+: no equivalent yet. Use wires. 
\item random \+: randtrigi 
\item sqrt \+: for now\+: use log-\/$>$div2-\/$>$+c-\/$>$exp. Adjust c to 32 to get the square root of the fraction mapped to 0..1 range. Adjust c to 8 to get the square root of the fraction mapped to 0..64 range. 
\item delay \+: delayedpulse 
\item pd \+: this is a subpatch object. Refer to the chapter \char`\"{}\+Subpatching\char`\"{}. 
\end{DoxyItemize}\hypertarget{md_pd_user_pd_user_idiom_conversion}{}\subsection{Idiom conversion}\label{md_pd_user_pd_user_idiom_conversion}
Equivalent patches from the book \char`\"{}\+Designing Sound\char`\"{} are in the directory patches/idioms.

\subsubsection*{Constrained counting}

Use the counter object.

\subsubsection*{Accumulator}

Will add accumulator object

\subsubsection*{Rounding}

Use the round object. Or force conversion of a fractional into an integer with the to\+Int object.

\subsubsection*{Scaling}

The mix1, mix2, mix3... objects can be used to scale and add one or more inputs.

\subsubsection*{Looping with until}

Mostly used to initialize tables. In Axoloti, tables can be initialized with C-\/code. For an example of this, see patches/tests/table.

\subsubsection*{Message complement}

For reciprocal, you can use log-\/$>$+64-\/$>$inv-\/$>$exp. Accurate to about 8 bits of precision.

\subsubsection*{Weighted Random Selection}

\subsubsection*{Delay cascade}

Use the delayedpulse object for generating delayed triggers.

\subsubsection*{Last Float and Averages}

\subsubsection*{Float Low Pass}

Use the \char`\"{}smooth\char`\"{} object. 